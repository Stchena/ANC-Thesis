\chapter{Walidacja rozwiązania}
\label{cha:tests}
Po wstępnym nastrojeniu układu, należy uruchomić oraz przetestować urządzenie. Autor zapoznał się w~tym celu z~normą dotyczącą mierzenia poziomu natężenia hałasu \cite{test_norm}, jednak ze względów subiektywnych nie zastosował się do niej, a~obrał własną metodę wykonania testu. Ze względu na niskie amplitudy sygnałów występujących w~urządzeniu i~podczas testu, autor zdecydował się przetestować urządzenie akwirując sygnały przy użyciu oscyloskopu. Zarówno sposób, jak i~wyniki pomiarów, opisane są w~sekcji \ref{sec:practical_test}.
\section{Test praktyczny}
\label{sec:practical_test}
Celem sprawdzenia jakości zbudowanego urządzenia, dokonano testu praktycznego, gdzie w~pomieszczeniu laboratoryjnym zmierzono poziom natężenia hałasu słyszanego przez mikrofony urządzenia. Do generowania dźwięków testowych użyto głośnika podłączonego do generatora funkcyjnego. Pomiaru dokonano w~trzech wariantach -- bez tłumienia, tylko z~tłumieniem pasywnym, oraz z~tłumieniem pasywnym i~uruchomionym tłumieniem aktywnym pochodzącym z~zaprogramowanego układu. W~przypadku drugiego testu udaje się wyznaczyć charakterystykę częstotliwościową pasywnej osłony dźwiękoszczelnej użytych w~projekcie nauszników. Z~kolei w~ostatnim teście, po pewnym czasie działania algorytmu, zostają odczytane wagi filtra LMS, co pozwala na wyznaczenie jego odpowiedzi impulsowej.
\subsection{Warunki testu}
\label{subsec:circumstances}
Układ został przetestowany w~pomieszczeniu przeznaczonym dla prowadzenia zajęć laboratoryjnych. W~trakcie przeprowadzania testu nie występowały inne, znaczące źródła hałasu, które mogłyby w~niekontrolowany sposób zakłócać pracę układu i~pomiar. Głośnik użyty do odgrywania dźwięków podłączono do generatora funkcyjnego i~umieszczono go w~bliskiej odległości od mikrofonu głównego zbudowanego urządzenia. Odległość od źródła dźwięku wynosiła \SI{45}{\mm}, zaś czas pomiaru wynosił 15~sekund dla każdego z~etapów.
\subsection{Typowe rodzaje hałasu}
Jak już wspomniano w~sekcji \ref{sec:hałas} rozdziału \ref{cha:teoria}, typowymi rodzajami hałasu, które można chcieć tłumić takim urządzeniem, są:
\begin{itemize}
	\item rozmowy pobliskich osób,
	\item szum wentylacyjny,
	\item dźwięk towarzyszący pracy silnika.
\end{itemize}
Są to zatem sygnały charakteryzujące się (w~przybliżeniu) stałym przedziałem częstotliwościowym i~niewielkim natężeniem, jednak będące uciążliwymi w~dłuższej perspektywie czasowej. W~ramach testu praktycznego sprawdzono działanie układu przy ekspozycji na hałas pochodzący od sygnałów sinusoidalnych, złożeń kilku takich sygnałów (np. sygnał sinusoidalny o~częstotliwości \SI{200}{\Hz} + \SI{700}{\Hz}), białego szumu oraz sygnału z~wobulatora będącego ,,przemieceniem" wszystkich częstotliwości w~zakresie pracy filtra.
\subsection{Poziom hałasu bez tłumienia}
Pomiaru bez tłumienia dokonano poprzez akwizycję sygnału z~mikrofonu głównego (feedforward). Pozwoliło to na zebranie danych o~poziomie natężenia hałasu przed jego przejściem przez osłonę dźwiękoszczelną. Przy założeniu opisanych w~sekcji \ref{subsec:circumstances} warunków testu, średni poziom hałasu zmierzony bez żadnego sposobu tłumienia wyniósł PLACEHOLDER. %TODO uzupełnić pomiar.
Widmo częstotliwościowe pokazano na rysunku \ref{fig:widmo_bez}.
\begin{figure}[h!]
	\centering
	\includegraphics{../Assets/widmo_bez_tlumienia.png}
	\caption{Widmo częstotliwościowe hałasu przy pomiarze bez tłumienia hałasu.}
	\label{fig:widmo_bez}
\end{figure}

\subsection{Poziom hałasu z pasywnym tłumieniem}
Następnie, aby dokonać pomiaru natężenia hałasu przy pasywnym tłumieniu, zamknięto i~zebrano dane pochodzące z~drugiego mikrofonu (feedback). Dzięki temu otrzymano informację o~tym, jaki poziom hałasu panuje wewnątrz słuchawki, czyli jaki hałas słyszałby użytkownik przy użyciu jedynie pasywnego tłumienia, które dają użyte nauszniki. Należy pamiętać, że zmieni się w~ten sposób również odległość od źródła hałasu do mikrofonu pomiarowego. Różnicę tę można jednak zawrzeć w~efektach działania tłumienia pasywnego -- odległość od źródła dźwięku również jest jedną ze składowych podejścia pasywnego. Przy tych samych założeniach, średni poziom hałasu wyniósł PLACEHOLDER2. %TODO uzupełnić pomiar.
Widmo częstotliwościowe zaprezentowano na rysunku \ref{fig:widmo_pasywnie}. 
\begin{figure}[h!]
	\centering
	\includegraphics{../Assets/widmo_pasywnie.png}	
	\caption{Widmo częstotliwościowe hałasu przy pomiarze z~pasywnym tłumieniem.}
	\label{fig:widmo_pasywnie}
\end{figure}

Dodatkowo, porównując wynik pomiaru bez tłumienia hałasu z~wynikiem otrzymanym w~tej sekcji, można wyznaczyć charakterystykę częstotliwościową osłony dźwiękoszczelnej. Aby informacja była miarodajna, należy posłużyć się do tego celu wobulatorem, który zapewnia sygnał dźwiękowy o~narastającej częstotliwości i~stałej amplitudzie. Taki rodzaj sygnału ujawnia szerokość pasma, które efektywnie tłumione jest przez samą osłonę o~wysokim współczynniku tłumienia. Wynik pomiaru, umieszczony na rysunku \ref{fig:sweep} pozwala lepiej zapoznać się z~parametrami akustycznymi użytych słuchawek i~w~dalszych iteracjach projektu usprawnić strojenie algorytmu tak, aby pasował do tego typu słuchawek.
\begin{figure}[h!]
	\centering
	\includegraphics{../Assets/widmo_sweep.png}	
	\caption{Widmo częstotliwościowe hałasu przy pomiarze badającym możliwości osłony pasywnej.}
	\label{fig:sweep}
\end{figure}
\subsection{Poziom hałasu z aktywnym tłumieniem}
Ostatecznie, aby dowieść skuteczności rozwiązania, uruchomiono algorytm i~ponownie zebrano dane z~mikrofonu odsłuchowego. Średni poziom hałasu zmierzony w~ten sposób wyniósł PLACEHOLDER3. %TODO uzupełnić pomiar.
Widmo częstotliwościowe zamieszczono na rysunku \ref{fig:widmo_aktywnie}.
\begin{figure}[h!]
	\centering
	\includegraphics{../Assets/widmo_aktywnie.png}	
	\caption{Widmo częstotliwościowe hałasu przy pomiarze z~aktywnym tłumieniem.}
	\label{fig:widmo_aktywnie}
\end{figure}

Na końcu tak wykonanego pomiaru odczytano wartości wag filtra LMS. Te nastawy zostały osiągnięte przez filtr w~całości samodzielnie, gdyż nie ustawiano żadnych wartości początkowych -- filtr był zerowany przy każdorazowym uruchomieniu. Taki pomiar pozwala na wyznaczenie odpowiedzi impulsowej zastosowanego filtra. Wizualizacja tych danych w~pakiecie obliczeniowym MATLAB pozwala na lepsze zrozumienie zachowania oraz możliwości adaptacyjnych filtra.
%TODO mikrofon z peceta do przestrzeni, zmierzyc widmo dzwieku do matlaba