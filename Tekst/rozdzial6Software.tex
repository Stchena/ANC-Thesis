\chapter{Software -- implementacja}
\label{cha:software}
W~tym rozdziale autor przedstawia sposób, w~jaki zaprogramowany został system. Uwzględnione oraz pokazane są tutaj różnice między typowym algorytmem FIR+LMS wspomnianym w~sekcji \ref{FIRLMS} rozdziału \ref{cha:teoria}, a~zaproponowanym przez autora wariantem rozwiązania i~charakterystycznym dla niego sposobem przetwarzania danych.
 
\section{Schemat przetwarzania danych}
Schemat rozwiązania autora jest niemalże równoważny schematowi typowego algorytmu, różni się jednak kolejnością i~charakterem pewnych działań. We~wspomnianym wcześniej wariancie błąd obliczany/mierzony jest na podstawie sygnału wyjściowego filtra FIR, tworzonego z~kolei na podstawie sygnału wejściowego oraz wartości współczynników w~danej chwili czasu. Natomiast w~wersji zaprojektowanej przez autora pomiar błędu następuje w~tej samej chwili, co pomiar sygnału wejściowego. Oznacza to zatem, że należy nieco zmienić kolejność wykonywanych czynności, co ukazuje poniższa tabela.
\begin{table}[h]
	\centering
	\caption{Porównanie schematów przetwarzania danych dla wspomnianego wcześniej algorytmu oraz wersji użytej przez autora.}
	\begin{tabular}{|p{.45\textwidth}|p{.45\textwidth}|}
		\toprule Klasyczny algorytm filtracji & Implementacja autora \\ \midrule
		\begin{enumerate}	
			\item Odczyt sygnału wejściowego x(n).
			\item Obliczenie sygnału wyjściowego y(n).
			\item Odczyt sygnału odsłuchowego e(n).
			\item Aktualizacja wartości współczynników filtra w(n).
		\end{enumerate} & 
		\begin{enumerate}	
			\item Odczyt sygnału wejściowego x(n) oraz sygnału odsłuchowego e(n).
			\item Aktualizacja wartości współczynników filtra w(n).
			\item Obliczenie sygnału wyjściowego y(n).
		\end{enumerate}\\ \bottomrule
	\end{tabular}
\end{table}

Kluczowe jest zatem zrozumienie, że w~n-tej chwili czasu mierzona jest wartość sygnału odsłuchowego odpowiadająca wartościom z~poprzedniej chwili czasu. Dlatego więc należy zaraz po odczytaniu zaktualizować wagi filtra, by zawsze obliczać sygnał wyjściowy na podstawie aktualnych danych.
\section{Implementacja algorytmu}
Algorytm został zaprogramowany w~języku~C przy użyciu środowiska System Workbench for STM32. Typ zmiennych użytych do obliczeń to liczby zmiennoprzecinkowe pojedynczej precyzji ,,float''. Aby skorzystać ze zwiększonej dokładności, jaką daje ten typ zmiennej, autor skonwertował zmienną przechowującą słowo bitowe (wynik pomiaru przetwornika) z~typu stałoprzecinkowego na zmiennoprzecinkowy, stosując w~pewnym sensie odwrócenie konwersji przetwornika. Słowo bitowe zostało przekonwertowane na poziom napięcia, od którego następnie odjęto wartość $ \frac{V_{ref}}{2} $, aby z~sygnału usunąć składową stałą. Jest to działanie konieczne dla poprawnego wykonywania operacji zgodnych z~algorytmem filtra. Po aktualizacji wag oraz obliczeniu sygnału wyjściowego, z powrotem dokonywana jest konwersja do słowa bitowego. Można więc powiedzieć, że programowo realizowany jest dodatkowy tor przetwarzania sygnału z~cyfrowego na pseudoanalogowy i~odwrotnie. Straty dokładności pochodzące z~tej zaprogramowanej konwersji są jednak znacznie mniejsze w~porównaniu do strat i~nakładu pracy pochodzącej z~wykonywania tych samych operacji na liczbach stałoprzecinkowych. Aby dodatkowo poprawić precyzję odpowiedzi filtra, zastosowano korekcję składowej, która pojawia sie w~wagach filtra FIR obliczanych algorytmem LMS -- w~każdej chwili czasowej przeliczana jest suma wag, a~gdy jest niezerowa, to jej część zostaje odjęta od każdej z~wag. Powoduje to powolne centrowanie nastaw algorytmu. Przyczyną dryfowania wag jest nieidealne wyeliminowanie składowych stałych z~sygnałów pochodzących z mikrofonów.

Na miejsce wykonania algorytmu FIR+LMS wybrano funkcję obsługującą przerwanie pochodzące od kontrolera DMA, informujące o~zakończonym transferze danych z~obu przetworników~ADC. Zostało to osiągnięte poprzez nadpisanie wskaźnika funkcji obsługi przerwania (,,callback''). W~jej wnętrzu wykonywany jest cały kod filtra oraz ustawienie wartości w~rejestrze przechowującym dane dla przetwornika~DAC.

Ponadto w~innych miejscach do tego przeznaczonych (głównie są to funkcje obsługujące różne przerwania) umieszczono kod odpowiadający za odbieranie lub wysyłanie danych UART.
\section{Dobór nastaw algorytmu}
Ponieważ filtr ma działać samodzielnie od momentu włączenia urządzenia, wagi można zainicjować wartościami zerowymi. Po pewnym czasie, na podstawie sygnału akustycznego podawanego na wejście układu, filtr będzie w~stanie obliczyć optymalne nastawy i~w~pewnym stopniu wytłumić hałas.

Parametr $\beta$ ustawiono metodą inżynierską na wartość 0.0002. Mniejsze wartości powodują wolniejszą reakcję filtra, zaś większe przesterowują go i,~choć mogą przyspieszyć i~usprawnić okres strojenia filtra, mogą też spowodować jego zdestabilizowanie.

Pamiętając o~przesterowaniach buforu przetwornika DAC w~pobliżu granicznych wartości, programowo ustawiono górny i~dolny próg słów bitowych wysyłanych na koniec algorytmu filtra. W~krańcowych przypadkach destabilizacji filtra, powoduje to powstawanie ograniczonej fali prostokątnej, która przez swoje charakterystyczne brzmienie dźwiękowo ostrzega testera o~zdestabilizowaniu filtra. Dodatkowo zabezpiecza to głośnik przed uszkodzeniem. Wadą takiego rozwiązania jest obniżenie rozdzielczości, co ma negatywny wpływ na jakość wysyłanego sygnału wyjściowego.