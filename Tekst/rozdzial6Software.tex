\chapter{Software -- implementacja}
\label{cha:software}
W~tym rozdziale autor przedstawia sposób, w~jaki zaprogramowany został system. Uwzględnione oraz pokazane są tutaj różnice między typowym algorytmem FIR+LMS wspomnianym w~sekcji \ref{FIRLMS} rozdziału \ref{cha:teoria}, a~zaproponowanym przez autora wariantem rozwiązania i~charakterystycznym dla niego sposobem przetwarzania danych.
 
\section{Schemat przetwarzania danych}
Schemat rozwiązania autora jest niemalże równoważny schematowi typowego algorytmu, różni się jednak kolejnością i~charakterem pewnych działań. Podczas gdy we~wspomnianym wcześniej wariancie błąd obliczany/mierzony jest na podstawie sygnału wyjściowego, tworzonego z~kolei na podstawie sygnału wejściowego oraz wartości współczynników w~danej chwili czasu, to w~wersji zaprojektowanej przez autora pomiar błędu następuje w~tej samej chwili, co pomiar sygnału wejściowego. Oznacza to zatem, że należy nieco zmienić kolejność wykonywanych czynności, co ukazuje poniższa tabela.
\begin{table}[h]
	\centering
	\caption{Porównanie schematów przetwarzania danych dla wspomnianego wcześniej algorytmu oraz wersji użytej przez autora.}
	\begin{tabular}{|p{.45\textwidth}|p{.45\textwidth}|}
		\toprule Klasyczny algorytm filtracji & Implementacja autora \\ \midrule
		\begin{enumerate}	
			\item Odczyt sygnału wejściowego x(n).
			\item Obliczenie sygnału wyjściowego y(n).
			\item Odczyt sygnału odsłuchowego e(n).
			\item Aktualizacja wartości współczynników filtra w(n).
		\end{enumerate} & 
		\begin{enumerate}	
			\item Odczyt sygnału wejściowego x(n) oraz sygnału odsłuchowego e(n).
			\item Aktualizacja wartości współczynników filtra w(n).
			\item Obliczenie sygnału wyjściowego y(n).
		\end{enumerate}\\ \bottomrule
	\end{tabular}
\end{table}

Kluczowe jest zatem zrozumienie, że w~n-tej chwili czasu mierzona jest wartość sygnału odsłuchowego odpowiadająca wartościom z~poprzedniej chwili czasu. Dlatego więc należy zaraz po odczytaniu zaktualizować wagi filtra, by zawsze obliczać sygnał wyjściowy na podstawie aktualnych danych.
\section{Implementacja algorytmu}
Algorytm został zaprogramowany w~języku~C przy użyciu środowiska System Workbench for STM32. Użyty typ zmiennych realizujących obliczenia to liczby zmiennoprzecinkowe pojedynczej precyzji ''float''. Aby skorzystać ze zwiększonej dokładności, jaką daje ten typ zmiennej, autor skonwertował zmienną przechowującą słowo bitowe (wynik pomiaru przetwornika) z~typu stałoprzecinkowego na zmiennoprzecinkowy, stosując w~pewnym sensie odwrócenie konwersji przetwornika. Słowo bitowe zostało przekonwertowane na poziom napięcia, od którego następnie odjęto $ \frac{V_{ref}}{2} $, aby z~sygnału usunąć składową stałą. Jest to działanie konieczne dla poprawnego wykonywania operacji zgodnych z~algorytmem filtra. Po aktualizacji wag oraz obliczeniu sygnału wyjściowego, z powrotem dokonywana jest konwersja do słowa bitowego. Można więc powiedzieć, że programowo realizowany jest dodatkowy tor przetwarzania sygnału z~cyfrowego na pseudoanalogowy i~odwrotnie. Straty dokładności pochodzące z~tej zaprogramowanej konwersji są jednak znacznie mniejsze w~porównaniu do strat i~nakładu pracy pochodzącej z~wykonywania tych samych operacji na liczbach stałoprzecinkowych.

Miejsce wykonania algorytmu umieszczono w~funkcji obsługującej przerwanie pochodzące od kontrolera DMA, informujące o~skończonym transferze danych z~obu przetworników~ADC.
\section{Dobór nastaw algorytmu}