\chapter{Analiza możliwych rozwiązań}
\label{cha:możliwe_układy}
 W~poprzednim rozdziale autor pokazał możliwe konstrukcje budujące układ tłumienia hałasu -- pasywny lub aktywny. W~tym rozdziale, autor wybiera konstrukcję hybrydową (pasywno-aktywną) feedforward-feedback oraz przedstawi możliwe do zastosowania platformy sprzętowe. Na podstawie analizy słabych i~mocnych stron wymienionych podejść, dokonany zostanie dalszy wybór.

 Zasada konstrukcyjna systemu aktywnego tłumienia hałasu w~żadnym kroku nie specyfikuje układu elektronicznego, na którym można zaimplementować algorytm tłumienia. Oznacza to zatem, że jeżeli konstruktor zastosuje się do praw fizyki i~wymagań projektu, to powinien być w~stanie zaimplementować cały system na dowolnym układzie -- analogowym lub cyfrowym. W~następnych sekcjach autor postara się odpowiedzieć na pytanie -- który z~nich najlepiej wybrać, a~jeśli cyfrowy, to jaka platforma sprzętowa jest najlepsza? 
\section{Układ analogowy}
\label{sec:analog}
Układ analogowy jest najtańszy i~najszybszy %TODO czy jest najprostszy?
 -- bazuje na wzmacniaczach operacyjnych odwracających i~odpowiednim wysterowaniu fazy sygnału. Daje możliwość osiągnięcia wysokich zakresów częstotliwościowych oraz charakteryzuje się bardzo niskim poborem prądu -- co sprzyja dobrej praktyce projektowania układów tanich w~zasilaniu. Pojawiają się jednak problemy przy implementowaniu pętli automatyki adaptacyjnego filtru -- złożoność układu znacznie wzrasta, ciężko jest też zdebuggować taki system, jeśli gdzieś popełniono błąd. W~zasadzie nie da się takiego rozwiązania zaprojektować inkrementacyjnie -- należy opracować od razu cały, kompletny system.%TODO Czy aby na pewno?
 Choć rozwiązanie jest mniej kosztowne i szybsze od innych, jest też zadziwiająco nieelastyczne, a~przy wzrastającej złożoności filtra znacząco komplikuje się jego konstrukcja.
\section{Układ cyfrowy}
\label{sec:digital}
Zastosowanie cyfrowego układu jest droższe i~wydaje się nieco trudniejsze, jednak zwiększa możliwości konstrukcyjne. Do wyboru jest kilka platform, bardzo często integrujących w~sobie sporo funkcjonalności, które w~podejściu tworzenia analogowego układu należałoby samodzielnie stworzyć (zakładając oczywiście, że technika analogowa na to pozwoli).  Pomimo początkowego zwiększenia złożoności układu, okazuje się, że w~dłuższej perspektywie czasowej ułatwione jest prototypowanie i~programowanie układu, ponieważ istnieje mnogość narzędzi inżynierskich pomagających w~procesie pracy z~układami cyfrowymi. Można nawet powiedzieć, że ucyfrowienie platformy realizującej tłumienie hałasu daje szansę na zaimplementowanie wymiennych algorytmów, filtrów i~dodanie kilku funkcjonalności, czyli uczynienie systemu modułowym i~przystosowanym do różnych zastosowań. Taka jest zresztą zasada działania kart dźwiękowych, które posiadają zaawansowane silniki efektów, dodawane, usuwane i~modyfikowane w~swobodny sposób.  W~odróżnieniu od ścieżki analogowej, układ taki jest też zdecydowanie mniej podatny na szumy i~przesłuchy sieci ze względu na charakter przesyłanego sygnału. Następne sekcje przybliżą przykłady platform, na których dałoby się zrealizować układ aktywnego tłumienia hałasu.
\subsection{Mikrokontroler}
\label{uC}
Odpowiednio złożonym, a~jednocześnie niezbyt skomplikowanym układem jest mikrokontroler. \\
Jedną z~najbardziej popularnych obecnie grup mikrokontrolerów jest rodzina urządzeń STM, bazujących na mikroprocesorach ARM~Cortex. Warto nadmienić, że firma STMicroelectronics posiada w~asortymencie zarówno same mikrokontrolery, jak i~płytki prototypowe z~zamontowanymi gotowymi peryferiami. Ta platforma udostępnia dwa możliwe sposoby programowania:
\begin{itemize}
	\item baremetal programming -- czyli programowanie czystego sprzętu na zasadzie proceduralnej bez zainstalowanego systemu operacyjnego,
	\item RTOS\footnote{System operacyjny czasu rzeczywistego (ang. Real-Time Operating System)} -- czyli instalacja i~obsługa odwołań systemu posiadającego jądro czasu rzeczywistego i~wspierającego tego typu zastosowania.
\end{itemize}
\subsection{Mikrokomputer jednopłytkowy}
\label{mikrokomp}
%TODO dopisac ze wykonalem test na raspberry na ALSIE
Taka platforma jest o~jeden krok bardziej zaawansowana od mikrokontrolera -- zazwyczaj zawiera dużo układów peryferyjnych oraz daje możliwość zainstalowania pełnego systemu operacyjnego Linux, Windows i~tym podobnych. 
\subsection{Karta dźwiękowa}
\label{soundcard}
Podstawowa karta dźwiękowa zawiera 4 główne elementy:
\begin{enumerate}
	\item Wejścia i~wyjścia do komunikacji z~mikrofonami lub głośnikami (i~tym podobnym).
	\item Przetwornik analogowo-cyfrowy (dokonuje konwersji wejścia).
	\item Przetwornik cyfrowo-analogowy (dokonuje konwersji wyjścia).
	\item Interfejs PCI\footnote{Magistrala komunikacyjna (ang. Peripheral Component Interconnect) przyłączająca rozszerzenia do płyty głównej komputera} lub inny, zazwyczaj używany do komunikacji z~płytą główną komputera, do którego wpięta jest karta.
\end{enumerate}
Większość nowoczesnych kart dźwiękowych zawiera jednak znacznie więcej niż wymienione komponenty, aby zapewnić wsparcie dla zaawansowanych funkcjonalności, takich jak dźwięk stereofoniczny 3D, wsparcie dla MIDI\footnote{Cyfrowy interfejs instrumentów muzycznych (ang. Musical Instrument Digital Interface)}, dodatkowe efekty dźwiękowe i~wiele innych. W~skład takich elementów wchodzą na przykład DSP\footnote{Procesor sygnałowy (ang. Digital Signal Processor)} lub nawet zaawansowane silniki dźwiękowe, dedykowane do~dokonywania przetworzeń dźwięku w~dziedzinie czasu lub częstotliwości (jak na przykład dodanie echa, opóźnienia czasowego, miksowanie efektów lub filtracja). Takie silniki stosowane są na przykład w~kartach dźwiękowych firmy Creative, aby odciążyć procesor ogólnego przeznaczenia i~przyspieszyć konwersje.\\
Z~uwagi na przeznaczenie kart dźwiękowych oraz różnorodność oferowanych przez nie funkcji, wydawać by się mogło, że są one idealnymi kandydatami na platformę sprzętową do układu aktywnego tłumienia hałasu. Tak jednak nie jest ze względu na niedostępność tej platformy -- twórcy kart dźwiękowych raczej nie udostępniają żadnych możliwości ingerowania w~oprogramowanie, zostawiając jedynie możliwość tworzenia sterowników obsługujących efekt działania kart poprzez API\footnote{Interfejs programowania aplikacji (ang. Application Programming Interface)}. Oznacza to w~takim razie brak samodzielności tej platformy, czyli uzależnienie jej od innego sprzętu, co nie spełnia wymogów projektu.
\subsection{FPGA}
\label{FPGA}

\section{Wybór rozwiązania}
\label{sec:wybór}

\subsection{Odpowiedni dobór środowiska programistycznego}
\label{sec:IDE}
%TODO wywalic to do wczesniejszej czesci pracy przy mikrokontrolerach