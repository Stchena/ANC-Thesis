\chapter{Konstrukcja mechaniczna projektu}
\label{cha:constr}
W~tym rozdziale, po wybraniu rodzaju konstrukcji oraz platformy sprzętowej, autor przedstawia zastosowane komponenty oraz generalną konstrukcję mechaniczną projektu.
\section{Zastosowane komponenty}
\label{sec:komponenty}
Aby skonstruować kompletny system pasywno-aktywnego tłumienia hałasu, autor zbudował swój projekt na bazie prostych, komercyjnych nauszników tłumiących, stosowanych w~budownictwie lub przemyśle. Nauszniki te stanowią część tłumiącą pasywnie. Do stworzenia i~zaimplementowania części aktywnej układu zastosowano wymienione poniżej elementy analogowe oraz cyfrowe, przy czym elementy cyfrowe są wymienione każdy z~osobna pomimo bycia integralną częścią mikrokontrolera.
\begin{enumerate}
	\item Analogowe:
	\begin{itemize}
		\item 2x Pojemnościowy mikrofon elektretowy CMA-4544PF-W
		\item 2x Przedwzmacniacz mikrofonowy Adafruit MAX4466
		\item Wzmacniacz audio klasy D Adafruit PAM8302
		\item Głośnik MG15 0.1W, 8 Ohm 
	\end{itemize}
	\item Cyfrowe:
	\begin{itemize}
		\item 
	\end{itemize}
\end{enumerate}
\section{Użyte elementy pasywnej redukcji}
\label{sec:usedPNC}

\section{Schemat i konfiguracja układu}
\label{sec:config}

