\documentclass[11pt]{aghdpl}
% \documentclass[en,11pt]{aghdpl}  % praca w języku angielskim

% Lista wszystkich języków stanowiących języki pozycji bibliograficznych użytych w pracy.
% (Zgodnie z zasadami tworzenia bibliografii każda pozycja powinna zostać utworzona zgodnie z zasadami języka, w którym dana publikacja została napisana.)
\usepackage[english,polish]{babel}

% Użyj polskiego łamania wyrazów (zamiast domyślnego angielskiego).
\usepackage{polski}

\usepackage[utf8]{inputenc}

% dodatkowe pakiety

\usepackage{mathtools}
\usepackage{amsfonts}
\usepackage{amsmath}
\usepackage{amsthm}

% --- < bibliografia > ---

\usepackage[
style=numeric,
sorting=none,
%
% Zastosuj styl wpisu bibliograficznego właściwy językowi publikacji.
language=autobib,
autolang=other,
% Zapisuj datę dostępu do strony WWW w formacie RRRR-MM-DD.
urldate=iso8601,
% Nie dodawaj numerów stron, na których występuje cytowanie.
backref=false,
% Podawaj ISBN.
isbn=true,
% Nie podawaj URL-i, o ile nie jest to konieczne.
url=false,
%
% Ustawienia związane z polskimi normami dla bibliografii.
maxbibnames=3,
% Jeżeli używamy BibTeXa:
backend=bibtex
]{biblatex}

\usepackage{csquotes}
% Ponieważ `csquotes` nie posiada polskiego stylu, można skorzystać z mocno zbliżonego stylu chorwackiego.
\DeclareQuoteAlias{croatian}{polish}

\addbibresource{bibliografia.bib}

% Nie wyświetlaj wybranych pól.
%\AtEveryBibitem{\clearfield{note}}


% ------------------------
% --- < listingi > ---

% Użyj czcionki kroju Courier.
\usepackage{courier}

\usepackage{listings}
\lstloadlanguages{TeX}

\lstset{
	literate={ą}{{\k{a}}}1
           {ć}{{\'c}}1
           {ę}{{\k{e}}}1
           {ó}{{\'o}}1
           {ń}{{\'n}}1
           {ł}{{\l{}}}1
           {ś}{{\'s}}1
           {ź}{{\'z}}1
           {ż}{{\.z}}1
           {Ą}{{\k{A}}}1
           {Ć}{{\'C}}1
           {Ę}{{\k{E}}}1
           {Ó}{{\'O}}1
           {Ń}{{\'N}}1
           {Ł}{{\L{}}}1
           {Ś}{{\'S}}1
           {Ź}{{\'Z}}1
           {Ż}{{\.Z}}1,
	basicstyle=\footnotesize\ttfamily,
}

% ------------------------

\AtBeginDocument{
	\renewcommand{\tablename}{Tabela}
	\renewcommand{\figurename}{Rys.}
}

% ------------------------
% --- < tabele > ---

\usepackage{array}
\usepackage{tabularx}
\usepackage{multirow}
\usepackage{booktabs}
\usepackage{makecell}
\usepackage[flushleft]{threeparttable}

% defines the X column to use m (\parbox[c]) instead of p (`parbox[t]`)
\newcolumntype{C}[1]{>{\hsize=#1\hsize\centering\arraybackslash}X}


%---------------------------------------------------------------------------

\author{Szymon Szczęsny}
\shortauthor{Sz. Szczęsny}

%\titlePL{Przygotowanie bardzo długiej i pasjonującej pracy dyplomowej w~systemie~\LaTeX}
%\titleEN{Preparation of a very long and fascinating bachelor or master thesis in \LaTeX}

\titlePL{Przedprototyp układu aktywnej redukcji poziomu hałasu}
\titleEN{Active noise reduction system - Proof of Concept}


\shorttitlePL{Przedprototyp układu aktywnej redukcji poziomu hałasu} % skrócona wersja tytułu jeśli jest bardzo długi
\shorttitleEN{Active noise control - PoC}

\thesistype{Praca dyplomowa inżynierska}
%\thesistype{Master of Science Thesis}

\supervisor{dr inż. Andrzej Tutaj}
%\supervisor{Marcin Szpyrka PhD, DSc}

\degreeprogramme{Automatyka i Robotyka}
%\degreeprogramme{Computer Science}

\date{2019}

\department{Katedra Automatyki i Robotyki}
%\department{Department of Applied Computer Science}

\faculty{Wydział Elektrotechniki, Automatyki,\protect\\[-1mm] Informatyki i Inżynierii Biomedycznej}
%\faculty{Faculty of Electrical Engineering, Automatics, Computer Science and Biomedical Engineering}

\acknowledgements{Serdecznie dziękuję \dots tu ciąg dalszych podziękowań np. dla promotora, żony, sąsiada, żony sąsiada itp.}
% TODO uzupełnić acknowledge

\setlength{\cftsecnumwidth}{10mm}

%---------------------------------------------------------------------------
\setcounter{secnumdepth}{4}
\brokenpenalty=10000\relax

\begin{document}

\titlepages

% Ponowne zdefiniowanie stylu `plain`, aby usunąć numer strony z pierwszej strony spisu treści i poszczególnych rozdziałów.
\fancypagestyle{plain}
{
	% Usuń nagłówek i stopkę
	\fancyhf{}
	% Usuń linie.
	\renewcommand{\headrulewidth}{0pt}
	\renewcommand{\footrulewidth}{0pt}
}

\setcounter{tocdepth}{2}
\tableofcontents
\clearpage

\chapter{Wprowadzenie - cel i zakres pracy}
\label{cha:intro}

\section{Cele pracy}
\label{sec:celePracy}

\section{Zawartość pracy}
\label{sec:zawartoscPracy}
\chapter{Problem badawczy - Hałas i metody jego tłumienia}
\label{cha:teoria}

\section{Hałas}
\label{sec:hałas}

\section{Pasywne tłumienie hałasu}
\label{sec:PNC}

\section{Aktywne tłumienie hałasu}
\label{sec:ANC}

\subsection{Feedforward}
\label{feedforward}

\subsection{Feedback}
\label{feedback}

\subsection{Feedforward-Feedback - układ hybrydowy}
\label{hybrid}


\chapter{Analiza możliwych rozwiązań}
\label{cha:możliwe_układy}

\section{Układ analogowy}
\label{sec:analog}

\section{Układ cyfrowy}
\label{sec:digital}

\subsection{Mikrokontroler}
\label{uC}

\subsection{Mikrokomputer}
\label{mikrokomp}

\subsection{Karta dźwiękowa}
\label{soundcard}

\section{Wybór rozwiązania}
\label{sec:wybór}
\chapter{Hardware i konstrukcja - zastosowane komponenty, elementy pasywnej redukcji, schemat i połączenie układu}
\label{cha:hardware}

\section{Zastosowane komponenty}
\label{sec:komponenty}

\section{Użyte elementy pasywnej redukcji}
\label{sec:usedPNC}

\section{Schemat i konfiguracja układu}
\label{sec:config}

\section{Wygenerowanie kodu w języku C}
\label{sec:configGenerate}
\chapter{Software - wybór środowiska programistycznego oraz implementacja}
\label{cha:software}

\section{Odpowiedni dobór środowiska programistycznego}
\label{sec:IDE}

\section{Schemat przetwarzania danych}

\section{Implementacja algorytmu}
\chapter{Walidacja rozwiązania - testy, symulacje komputerowe a rzeczywistość, oczekiwane~rezultaty}
\label{cha:tests}

\section{Układ symulowany w środowisku Simulink}

\section{Test praktyczny}

\subsection{Warunki testu}
\subsection{Typowe rodzaje hałasu}
\subsection{Poziom hałasu bez tłumienia}
\subsection{Poziom hałasu z pasywnym tłumieniem}
\subsection{Poziom hałasu z aktywnym tłumieniem}
\subsection{Poziom hałasu przy użyciu obu technik}
\chapter{Podsumowanie i wnioski}
\label{cha:wnioski}

\section{Ocena jakości rozwiązania}

\section{Propozycja możliwych usprawnień}

\section{Wnioski ogólne}



% itd.
% \appendix
% \include{dodatekA}
% \include{dodatekB}
% itd.

\printbibliography

\end{document}
