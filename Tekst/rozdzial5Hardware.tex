\chapter{Hardware -- konfiguracja}
Ten rozdział poświęcony jest dokładnemu opisowi konfiguracji układu i~jego peryferiów oraz procesu generowania kodu w~C. Autor wyszczególnia kluczowe decyzje konfiguracyjne i~uzasadnia podjęte wybory. Autor przytacza fragmenty dokumentacji płytki ewaluacyjnej, instrukcji obsługi użytkownika oraz innych dokumentów dostarczonych przez producenta mikrokontrolera, aby poprzeć wywody i~obliczenia obecne w~tym rozdziale. Całość procesu dobierania parametrów i~ustawień komponentów płytki ewaluacyjnej przeprowadzona jest we wspomnianym w~sekcji \ref{sec:IDE} środowisku STM32CubeMX. Zastosowano podejście bare-metal programming, by w~jak największym stopniu ograniczyć narzuty czasowe operacji niezwiązanych ściśle z~przetwarzaniem dźwięku.
\section{Obliczeniowa platforma sprzętowa}

\section{Konfiguracja peryferiów}
Konfigurację systemu należy rozpocząć od uruchomienia wszystkich wymienionych w~poprzednim rozdziale elementów cyfrowych. Dobór parametrów omówiony zostanie zatem po kolei dla każdego komponentu:
\begin{itemize}
	\item Przetwornik analogowo-cyfrowy nr 1 (od głównego mikrofonu)\\
	Oba przetworniki skonfigurowane są w~trybie dualnym jednoczesnej konwersji regularnej\footnote{Ang. Dual regular simultaneous mode}. Pozwala to na synchronizowane działanie komponentów z~przetwornikiem numer~1~jako elementem nadrzędnym. Przetwarzany sygnał kierowany jest do kanału 0. 
\end{itemize}
\section{Wygenerowanie kodu w języku C}
\label{sec:configGenerate}