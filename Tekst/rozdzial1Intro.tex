\chapter{Wprowadzenie -- cel i zakres pracy}
\label{cha:intro}

\section{Cele pracy}
\label{sec:celePracy}
Celem pracy jest przybliżenie czytelnikowi problemu hałasu i~jego tłumienia oraz  zaprojektowanie kompletnego rozwiązania dla takiego zadania inżynierskiego. W~skład projektu wchodzi przeanalizowanie dostępnych platform sprzętowych, zaprojektowanie systemu, wykonanie, oprogramowanie, uruchomienie, dostrojenie oraz przetestowanie prototypowego układu tłumienia hałasu. Aby zmniejszyć koszty i~zmaksymalizować szanse powodzenia projektu, autor używa powszechnie dostępnych rynkowo komponentów i~dobrze przebadanego, pozwalającego na efektywną implementację algorytmu. Konfiguracyjny kod programu zostanie wygenerowany przy użyciu procedur automatycznej generacji kodu, zaś algorytm zostanie zaimplementowany ręcznie przy użyciu języka programowania wysokiego poziomu~C.\\
Praca ma również na celu sprawdzenie walorów rynkowych proponowanego przez autora rozwiązania poprzez porównanie z~istniejącym, powszechnie stosowanym produktem.
\subsection{Specyfikacja wymagań projektu}
Aby wybrane przez autora rozwiązanie można było traktować jako zadowalające i~rozwiązujące problem, musi ono posiadać następujące cechy:
\begin{enumerate}
	\item Koszt części, implementacji oraz zamontowania systemu nie powinien stanowić znaczącego udziału w~koszcie przykładowych słuchawek z~takim systemem -- innymi słowy, rozwiązanie powinno być względnie tanie.
	\item Układ musi działać samodzielnie, bez wsparcia innego sprzętu -- czyli w~całości ma realizować powierzone zadanie bez zewnętrznych sterowników ani potrzeby ręcznego strojenia po uruchomieniu.
	\item Układ musi być efektywny, a~różnica powodowana przez jego uruchomienie powinna być zauważalna co najmniej na podstawie testów sprzętu. Prototyp okaże się bardzo dobry, jeśli różnica będzie słyszalna gołym uchem.
	\item Prototyp nie musi umożliwiać słuchania muzyki przy jednoczesnym uruchomieniu tłumienia.
	\item System powinien łączyć w sobie zalety pasywnego oraz aktywnego tłumienia, aby w~pełni wykorzystać potencjał konstrukcji.
\end{enumerate}
\section{Zawartość pracy}
\label{sec:zawartoscPracy}
Rozdział I niniejszej pracy zawiera wprowadzenie, specyfikację wymagań projektu i~przegląd literatury ukierunkowany na umiejscowienie prezentowanego rozwiązania na tle obecnego stanu techniki, wraz z~wyszczególnieniem relacji łączących poszczególne pozycje z~zawartością pracy.

Z kolei rozdział II wprowadza czytelnika w~zagadnienia teoretyczne dotyczące hałasu i~istniejących obecnie metod jego redukowania. Przedstawia dostępne rozwiązania konstrukcyjne aktywnego tłumienia oraz pokrótce wyjaśnia zasadę działania adaptacyjnego algorytmu filtracyjnego.

Następnie, w~rozdziale III, autor na podstawie konstrukcji z~poprzedniego rozdziału opisze dostępne platformy sprzętowe, przeanalizuje ich wady i~zalety oraz dokona dalszego wyboru sposobu rozwiązania problemu. W~części poświęconej wyborowi podejścia autor przedstawi również dostępne środowiska służące do prototypowania oraz programowania układu.

Po dokonaniu wyboru, rozdział IV będzie wprowadzeniem do praktycznej części tej pracy, gdzie autor wymieni zastosowane komponenty, elementy pasywnej redukcji, przedstawi schemat i~konfigurację układu oraz opisze, w~jaki sposób wygenerował kod konfiguracyjny, który posłużył mu do zaimplementowania algorytmu filtracji.

W~dalszej części pracy, w~rozdziale V, zostanie bliżej opisana obliczeniowa platforma sprzętowa oraz połączone z~nią peryferia -- mikrofony, głośnik, oraz wymagane przedwzmacniacze sygnału audio. Autor na podstawie obliczeń zweryfikuje poprawność konfiguracji.

Następnym etapem pracy jest implementacja algorytmu, co zostanie przedstawione dokładnie w~rozdziale VI. Autor w~przystępny sposób przybliży schemat przetwarzania danych, opisze zastosowany algorytm, zaimplementuje go oraz dobierze jego początkowe nastawy.

Wreszcie, po stworzeniu kompletnego systemu, w~rozdziale VII zostanie on porównany z~symulacją komputerową Simulink oraz wykonany będzie test praktyczny w~środowisku podatnym na hałas.

Na końcu, w~rozdziale VIII autor podsumuje efekty projektu oceniając jego jakość w~porównaniu do istniejących rozwiązań komercyjnych, wyciągnie wnioski i~zaproponuje możliwe usprawnienia systemu.
\section{Przegląd literatury}
Problem aktywnego tłumienia hałasu jest znany już od 1933~roku, gdy niemiecki wynalazca Paul~Lueg złożył wniosek patentowy \cite{LuegPatent} dotyczący procesu wyciszania oscylacji dźwiękowych poprzez przesunięcie fazy sygnału tak, aby pod wpływem superpozycji faz dźwięku oryginalnego oraz tłumiącego, wywołać mechaniczne wytłumienie -- co w fizyce nazywa się interferencją destruktywną.

Obecnie, zagadnienie to jest dobrze przebadane i~stosowane dość powszechnie zarówno w~przemyśle, jak~i~urządzeniach konsumenckich. Istnieje wiele implementacji układów tłumienia hałasu, różniących się budową, kosztami oraz efektywnością. Autor, realizując zadanie inżynierskie, polegające na budowie układu tłumiącego, posiłkuje się przy wyborze i implementacji rozwiązania badaniami oraz zaleceniami z prac poruszających interesujące zagadnienie. Praca ta jest najbardziej zbliżona tematycznie do artykułu ''Active noise control system for headphone applications'' \cite{ANC4HP} autorstwa S.M.~Kuo, S.~Mitra, Woon-Seng~Gan opublikowanego w czasopiśmie ''IEEE Transactions on Control Systems Technology'', bowiem autor zdecydował się zrealizować zadanie przy użyciu filtru FIR\footnote{Filtr o skończonej odpowiedzi impulsowej (ang. Finite Impulse Response)} oraz  adaptacyjnego algorytmu LMS\footnote{%TODO TU NAZWA PO POLSKU
(ang. Least-Mean-Squares)}. Wybór sposobu zaimplementowania rozwiązania został jednak podjęty niezależnie od implementacji autorów artykułu, na podstawie analizy wspomnianych różnic pomiędzy platformami sprzętowymi.

Jako głównej bazy dla stworzenia filtru FIR z algorytmem LMS, autor użył książek ''Adaptive Filter Theory'' oraz ''Least-Mean-Square Adaptive Filters'' napisanych przez Simon'a~Haykin'a, powszechnie uznawanego za pioniera oraz autorytet w~sztuce adaptacyjnego przetwarzania sygnałów. Obie prace podają rozbudowaną teorię dotyczącą budowy takich układów z~uwzględnieniem funkcji przejścia i aspektów teorii sterowania. Ponieważ jednak sama implementacja nie jest jeszcze gotowym układem, należy zwrócić uwagę na bardzo ważny aspekt, jakim jest strojenie algorytmu.

Aby poprzeć tworzenie filtru oraz by zastosować się do powszechnie stosowanych i~uznawanych praktyk pracy z~systemami audio, autor wsparł się tytułem ''Principles of Digital Audio'' autorstwa Ken'a~Pohlmann'a, również uznawanego autorytetu w~dziedzinie akustyki oraz przetwarzania sygnałów. Pozycja ta zapoznaje czytelnika z~takimi pojęciami, jak numeryczna reprezentacja d\'zwięku, twierdzenie o~próbkowaniu Nyquista-Shannona, kodowanie w~dziedzinie częstotliwości oraz THD+N\footnote{Współczynnik zawartości harmonicznych + Szum	(ang. Total Harmonic Distortion + Noise)}. Pojęcia te są szczególnie ważne w~rozdziale poświęconym testom zbudowanego układu -- to~tam zostaną zmierzone najważniejsze parametry d\'zwiękowe uzyskanego sygnału, celem określenia przydatności rozwiązania.

Celem wybrania konkretnego podejścia, autor użył źródeł internetowych, własnych doświadczeń z~platformami sprzętowymi oraz wykonał podstawowe testy opóźnień czasowych sterowników dźwięku ALSA na mikrokomputerze Raspberry Pi Model 3B.

Wreszcie, po podjęciu decyzji odnośnie platformy, na której realizowany będzie cały system, autor użył dokumentacji, not katalogowych i~podręczników użytkownika dostarczanych przez producentów stosowanej platformy sprzętowej oraz używanych peryferiów (\cite{speakeropamp}, \cite{RM0390}). %TODO dodac tutaj w nawiasie pozycje z literatury
Przegląd wymienionych pozycji jest wręcz niezbędny, gdyż bez niego nie można dopasować urządzeń peryferyjnych (mikrofony, głośniki, wzmacniacze) bez narażania układu na uszkodzenie, lub w~najlepszym wypadku na bardzo niską jakość działania.\\
Podczas tworzenia symulacji komputerowych równie ważna okazała się dokumentacja pakietu oprogramowania do obliczeń numerycznych i~symulacji komputerowych Matlab/Simulink. 

Ostatecznie, podczas testów, autor bazował na normie ISO~3740:2019 \cite{test_norm} traktującej o~sposobach mierzenia poziomu natężenia dźwięku. Ponieważ test polegał na porównaniu działania systemu zbudowanego przez autora z~wybranym istniejącym, komercyjnym rozwiązaniem, skorzystano z~dokumentacji słuchawek Jabra~Evolve~80 \cite{JabraEvolve80}, w której można znaleźć podstawowe informacje o~zastosowanym w słuchawkach rozwiązaniu ANC (Active Noise Control -- Aktywne Tłumienie Hałasu).