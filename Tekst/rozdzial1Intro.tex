\chapter{Wprowadzenie -- cel i zakres pracy}
\label{cha:intro}

\section{Cele pracy}
\label{sec:celePracy}
Celem pracy jest przybliżenie czytelnikowi problemu hałasu i~jego tłumienia oraz  zaprojektowanie kompletnego rozwiązania dla takiego zadania inżynierskiego. W~skład projektu wchodzi przeanalizowanie dostępnych platform sprzętowych, zaprojektowanie systemu, wykonanie, oprogramowanie, uruchomienie, dostrojenie oraz przetestowanie prototypowego układu tłumienia hałasu. Aby zmniejszyć koszta i~zmaksymalizować szanse powodzenia projektu, autor używa powszechnie dostępnych rynkowo komponentów i~dobrze przebadanego, względnie prostego w~implementacji algorytmu. Konfiguracyjny kod programu zostanie wygenerowany przy użyciu procedur automatycznej generacji kodu%TODO Sprawdzic czy CubeMX jest licencjonowany :P
, zaś algorytm zostanie zaimplementowany ręcznie przy użyciu języka programowania wysokiego poziomu~C.\\
Praca ma również na celu sprawdzenie walorów rynkowych proponowanego przez autora rozwiązania poprzez porównanie z~istniejącym, powszechnie stosowanym produktem.
%TODO specyfikacja wymagan projektu
\subsection{Specyfikacja wymagań projektu}

\section{Zawartość pracy}
\label{sec:zawartoscPracy}
Rozdział I niniejszej pracy zawiera wprowadzenie, specyfikację wymagań projektu i~przegląd literatury ukierunkowany na relację tego opracowania do obecnego stanu wiedzy wraz z~wyszczególnieniem relacji łączących poszczególne tytuły z~zawartością pracy. \\
Z kolei rozdział II wprowadza czytelnika w~informacje teoretyczne dotyczące hałasu i~istniejących obecnie metod jego redukowania.
\section{Przegląd literatury}
Problem aktywnego tłumienia hałasu jest znany już od 1933~roku, gdy niemiecki wynalazca Paul~Lueg złożył wniosek patentowy\cite{LuegPatent} dotyczący procesu wyciszania oscylacji dźwiękowych poprzez przesunięcie fazy sygnału tak, aby pod wpływem superpozycji faz dźwięku oryginalnego oraz tłumiącego, wywołać mechaniczne wytłumienie -- co w fizyce nazywa się interferencją destruktywną.

Obecnie, zagadnienie to jest dobrze przebadane i~stosowane dość powszechnie zarówno w~przemyśle, jak~i~urządzeniach konsumenckich. Istnieje wiele implementacji układów tłumienia hałasu, różniących się budową, kosztami oraz efektywnością. Autor, realizując zadanie inżynierskie, polegające na budowie układu tłumiącego, posiłkuje się przy wyborze i implementacji rozwiązania badaniami oraz zaleceniami z prac poruszających zagadnienie wcześniejsze. Praca ta jest najbardziej zbliżona tematycznie do artykułu ''Active noise control system for headphone applications'' autorstwa S.M.~Kuo, S.~Mitra, Woon-Seng~Gan opublikowanego w czasopiśmie ''IEEE Transactions on Control Systems Technology'', bowiem autor zdecydował się zrealizować zadanie przy użyciu filtru FIR\footnote{Filtr o skończonej odpowiedzi impulsowej (ang. Finite Impulse Response)} oraz  adaptacyjnego algorytmu LMS\footnote{%TODO TU NAZWA PO POLSKU
(ang. Least-Mean-Squares)}. Wybór sposobu zaimplementowania rozwiązania został jednak podjęty niezależnie od implementacji autorów artykułu, na podstawie analizy wspomnianych różnic pomiędzy platformami sprzętowymi.

Jako głównej bazy dla stworzenia filtru FIR z algorytmem LMS, autor użył ''Adaptive Filter Theory'' oraz ''Least-Mean-Square Adaptive Filters'' napisanych przez Simon'a~Haykin'a, powszechnie uznawanego za pioniera oraz autorytet w~sztuce adaptacyjnego przetwarzania sygnałów. Obie prace podają rozbudowaną teorię dotyczącą budowy takich układów z~uwzględnieniem funkcji przejścia i aspektów teorii sterowania. Ponieważ jednak sama implementacja nie jest jeszcze gotowym układem, należy zwrócić uwagę na bardzo ważny aspekt, jakim jest strojenie algorytmu.

Aby poprzeć tworzenie filtru oraz by zastosować się do powszechnie stosowanych i~uznawanych praktyk pracy z~systemami audio, autor wsparł się tytułem ''Principles of Digital Audio'' autorstwa Ken'a~Pohlmann'a, również uznawanego autorytetu w~dziedzinie akustyki oraz przetwarzania sygnałów. Pozycja ta zapoznaje czytelnika z~takimi pojęciami, jak numeryczna reprezentacja d\'zwięku, twierdzenie o~próbkowaniu Nyquista-Shannona, kodowanie w~dziedzinie częstotliwości oraz THD+N\footnote{Współczynnik zawartości harmonicznych + Szum	(ang. Total Harmonic Distortion + Noise)}. Pojęcia te są szczególnie ważne w~rozdziale poświęconym testom zbudowanego układu -- to~tam zostaną zmierzone najważniejsze parametry d\'zwiękowe uzyskanego sygnału, celem określenia przydatności rozwiązania.

Celem wybrania konkretnego podejścia, autor użył PLACEHOLDER %TODO dodać zrodla
oraz wykonał podstawowe testy opó\'znień czasowych sterowników d\'zwięku ALSA na mikrokomputerze Raspberry Pi Model 3B oraz PLACEHOLDER2.%TODO dodac cos jeszcze.

Wreszcie, po podjęciu decyzji odnośnie platformy, na której realizowany będzie cały system, autor użył dokumentacji, not katalogowych i~manuali dostarczanych przez producentów stosowanej platformy sprzętowej oraz używanych peryferiów. Przegląd wymienionych pozycji jest wręcz niezbędny, gdyż bez niego nie można dopasować urządzeń peryferyjnych (mikrofony, głośniki, wzmacniacze) bez narażania układu na uszkodzenie, lub w~najlepszym wypadku na bardzo niską jakość działania.\\
Podczas tworzenia symulacji komputerowych równie ważna okazała się dokumentacja Matlab/Simulink. 

Ostatecznie, podczas testów, autor bazował na normie ISO~3740:2019\footnote{https://www.iso.org/obp/ui/\#iso:std:iso:3740:ed-3:v1:en} traktującej o~sposobach mierzenia poziomu natężenia d\'zwięku. Ponieważ test polegał na porównaniu działania systemu zbudowanego przez autora z~wybranym istniejącym, komercyjnym rozwiązaniem, użyto manuala słuchawek Jabra~Evolve~80 \cite{JabraEvolve80}, w którym można znale\'zć podstawowe informacje o~zastosowanym w słuchawkach rozwiązaniu ANC.