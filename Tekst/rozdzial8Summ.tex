\chapter{Podsumowanie i wnioski}
\label{cha:wnioski}
W~tym rozdziale autor podsumowuje wyniki pomiarów z~poprzedniego rozdziału, ocenia jakość rozwiązania względem istniejących rozwiązań komercyjnych oraz proponuje możliwe usprawnienia swojego projektu.
\section{Ocena jakości rozwiązania}
Koszt zbudowanego urządzenia, po podliczeniu kosztów wszystkich komponentów zamówionych w~sklepie internetowym Botland oraz kosztu nauszników ochronnych, zamyka się w~granicach~220~zł. Poszczególne elementy składające się na tę kwotę znajdują się w~tabeli \ref{tab:costs}.
\begin{table}[h!]
	\centering
	\caption{Wyliczenie kosztów poszczególnych elementów składowych projektu.}
	\label{tab:costs}
		\begin{tabular}{|p{.25\textwidth} | p{.35\textwidth} | p{.10\textwidth}|}
		\hline 
		Element & Opis & Koszt \\ 
		\hline\hline
		NUCLEO-F446RE & Płytka prototypowa z~mikrokontrolerem. & 77,90 zł \\ 
		\hline
		Adafruit MAX4466 -- dwie sztuki & Mikrofon elektretowy z przedwzmacniaczem audio. & 79,80 zł \\
		\hline
		Adafruit PAM8302 & Wzmacniacz audio dedykowany do głośników. & 19,90 zł \\
		\hline
		Głośnik MG15 & Głośnik \SI{0,1}{\W} o~impedancji \SI{8}{\Omega}. & 2,20 zł \\
		\hline
		Nauszniki ochronne & Typowe, tanie nauszniki ochronne ze sklepu BHP. & 12,15 zł \\
		\hline
		Zestaw płytka stykowa + przewody + moduł zasilający & Podstawowy zestaw płytki stykowej 830 pól, przewody oraz moduł zasilający MB102. & 19,90 zł \\
		\hline
	\end{tabular}
\end{table}

Zatem, według tabeli \ref{tab:costs}, sumaryczny koszt takiego rozwiązania wynosi 211,85 zł. Należy pamiętać, że jest to kwota brutto za zamówienie małej liczby elementów -- przy zamówieniu hurtowym części do masowej produkcji urządzenia, cena byłaby zapewne istotnie niższa.

W~porównaniu do słuchawek Jabra Elite 80, które można kupić w~cenie wahającej się między 1000~zł, a~1400~zł\footnote{Wyniki wyszukiwania serwisu ceneo.pl, stan z~dnia 04.01.2020.}, jest to dobra cena, jeśli weźmie się pod uwagę fakt, że dokupienie dodatkowych elementów i~zmodyfikowanie układu tak, by spełniał funkcję słuchawek, nie zwiększy jego ceny nawet dwukrotnie.

Ponieważ jednak nie udaje się efektywnie wytłumić większej ilości złożonych rodzajów hałasu w~zmiennych warunkach, prototypowe urządzenie opracowane w~tym projekcie nie może w~pełni równać się z~gotowym, dopracowanym rozwiązaniem stosowanym w~komercyjnym produkcie. 
\section{Propozycja możliwych usprawnień}
Biorąc pod uwagę decyzje konstrukcyjne, ograniczenia użytej platformy oraz wyniki testów, autor proponuje następujące usprawnienia swojego projektu:
\begin{enumerate}
	\item Użycie dedykowanej płytki z~mikrokontrolerem, zamiast płytki prototypowej, która zawiera niepotrzebne w~tym układzie elementy.\\
	Pozwoliłoby to na obniżenie kosztów jednostki centralnej i~zmniejszenie wymiarów urządzenia -- przy użyciu dedykowanej płytki, można by od razu poprowadzić na niej połączenia, które w~obecnym stanie realizowane są poprzez płytkę stykową.
	\item Dokładniejsze zamocowanie elementów w~słuchawce.\\
	Obecnie, mikrofon główny, odsłuchowy i~głośnik połączone są dość luźno z~nausznikami. Urządzenie działałoby na pewno sprawniej, gdyby przytwierdzono te elementy na stałe i~dostosowano ich rozmiar oraz otoczenie do współpracy, gdyż obecnie są to niezwiązane ze sobą części, połączone do celów prototypowych.
	\item Rozszerzenie funkcjonalności urządzenia o~drugi kanał.\\
	Dodanie kanału stereo, drugiego głośnika i~drugiego mikrofonu odsłuchowego pozwoliłoby na pełnoprawne korzystanie z~nauszników -- w~tym momencie jest to jedynie odgrodzona sferyczna przestrzeń, którą tworzą dwie złączone słuchawki.
	\item Dodanie możliwości słuchania pożądanego sygnału w~urządzeniu.\\
	Urządzenie obecnie jedynie tłumi sygnał -- mogłoby jednak tłumić hałas, a~jednocześnie odtwarzać muzykę lub inne pożądane dźwięki, tak jak robią to słuchawki komercyjne stosowane na rynku.
	\item Ujednolicenie zasilania.\\
	W~tym momencie cały system zasilany jest z~trzech źródeł:
	\begin{itemize}
	\item Zasilanie USB doprowadzone do płytki prototypowej NUCLEO-F446RE,
	\item Zasilanie USB doprowadzone do modułu zasilającego przedwzmacniacz audio,
	\item Zasilanie bateryjne dostarczające energię do mikrofonów urządzenia.
	\end{itemize}
	Ograniczenie liczby połączeń układu do jednego ułatwiłoby używanie urządzenia i~zwiększyłoby jego mobilność.
	\item Dodanie algorytmu estymacji i~eliminacji składowej stałej z~sygnałów pochodzących z~przedwzmacniaczy mikrofonowych.\\
	Dla zastosowanego układu mikrofonu z~przedwzmacniaczem składowa stała jest równa połowie napięcia zasilania. Oznacza to, że przy zasilaniu bateryjnym, wraz z~charakterystycznym dla baterii spadkiem napięcia w~miarę wyładowywania, spadać będzie również wartość tej składowej. Jeśli więc ustawi się ją na pewną stałą wartość, tak jak ma to miejsce w~obecnej wersji programu, to filtr będzie tracił dokładność z~powodu rozładowania baterii.
	\item Rozszerzenie zakresu częstotliwości w~którym efektywnie działa aktywne tłumienie szumów.\\
	Górną częstotliwość przedziału podniosłoby zwiększenie częstotliwości próbkowania, zaś dolną -- zwiększenie długości filtra. Dla celów prototypowych ustawiono mały zakres dostępnego tłumienia, jednak w~następnych iteracjach urządzenia, po zoptymalizowaniu konfiguracji i~programu, można by zwiększyć szerokość tego pasma.
	\item Zastosowanie bardziej rozbudowanych bądź zmodyfikowanych wersji algorytmu LMS.
	\item Wykorzystanie trybu obniżonego poboru energii przez mikrokontroler.\\
	Tryb, w~którym rdzeń usypia, ale pracują peryferia i~w~którym można go wybudzić przerwaniem, podniósłby efektywność urządzenia i~zmniejszyłby koszty jego eksploatacji.
	\item Optymalizacja kosztów urządzenia.
	\item Obniżenie częstotliwości taktowania poszczególnych podzespołów mikrokontrolera w~celu obniżenia pobory mocy.
	\item Dodanie łączności bezprzewodowej.\\
	Można  na przykład użyć modułu Bluetooth -- urządzenie spełniałoby wtedy funkcję słuchawek bezprzewodowych.
	\item Dodanie układu zarządzania akumulatorem.\\
	Układy BMS (Battery Management System) pomagają w~bardziej ekonomiczny sposób zużywać energię, co przekładałoby się na tańszą eksploatację urządzenia i~jego wydłużone działanie na jednym ładowaniu.
\end{enumerate}

Powyższe propozycje są jedynie przykładowymi sposobami ulepszenia działania lub dodania nowych funkcjonalności urządzenia.
\section{Wnioski ogólne}
Prototypowanie urządzeń przy użyciu mikrokontrolerów STM32 jest prostsze w~stosunku do projektowania układów analogowych lub cyfrowych nieużywających tego typu mikrokontrolera. Zapewnia to mnogość dokumentacji i~narzędzi dostarczanych na bieżąco przez producentów oprogramowania.

Bardzo często wykorzystuje się mikrokontrolery do zaprojektowania i~przetestowania układu, który później zostaje wykonany w~wersji analogowej dla zaoszczędzenia energii i~zmniejszenia wymiarów ostatecznej postaci urządzenia.


i tak dalej i tak dalej i żyli długo i szczęśliwie za siedmioma sinusami